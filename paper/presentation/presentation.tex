\documentclass{beamer}

\mode<presentation> {
	\usetheme{Darmstadt}
	\usecolortheme{beaver}
}

\usepackage{graphicx}
\usepackage{booktabs}
\usepackage{lmodern}

\setbeamertemplate{blocks}[rounded][shadow=false]
\setbeamerfont{bibliography entry author}{size=\tiny}%
\setbeamerfont{bibliography entry title}{size=\tiny}
\setbeamerfont{bibliography entry journal}{size=\tiny}
\setbeamertemplate{itemize items}[default]
\setbeamertemplate{enumerate items}[default]
\setbeamerfont{bibliography entry note}{size=\tiny}

\newtheorem{proposition}{Proposition}
\newtheorem{hypothesis}{Hypothesis}

\title[Dynamic Symmetry Breaking]{Branching Heuristics Promoting Symmetry Activity}

\author{
	J. van Beek \and
	I.J.G. in `t Veen \and
	A.J. Rouvoet \and
	E. Schoute
}
\institute[TU Delft]
{
	Delft, University of Technology \\
	\medskip
	\textit{\{j.vanBeek, a.j.Rouvoet, e.Schoute, i.j.g.intVeen\}@student.tudelft.nl}
}

\date{\today}

\begin{document}

	\begin{frame}
		\titlepage % Print the title page as the first slide
	\end{frame}

	\begin{frame}
		\frametitle{Overview}
		\tableofcontents
	\end{frame}

\section{Introduction}

	\begin{frame}
		\frametitle{Introduction}
		\begin{block}{Symmetry Propagation}
			A feasible dynamic symmetry breaking technique for SAT-solvers as proposed by
			\cite{devriendt2012symmetry} in 2012.
		\end{block}
	\end{frame}

	\begin{frame}
		\frametitle{Research questions}
		\begin{block}{Hypothesis 1}
			\label{hyp:increased_activity}
			Heuristic $H$ manages to keep more symmetries weakly active during search.
		\end{block}
		
		\visible<+->{
		\begin{block}{Hypothesis 2}
			\label{hyp:activity_equals_speed}
			When more symmetries are active during search, more literals can be propagated, resulting in
			better solve performance on average.
		\end{block}
		}
	\end{frame}

\section{Background}

	\subsection{SAT}
	\begin{frame}
		\frametitle{SAT notation}

		\begin{itemize}
			\item<1-> theory $T$ is a conjunction of clauses
			\item<1-> clause $c$ is a disjunction of literals
			\item<1-> literal $l$ is of the form $x$ or $\neg x$ and $x \in \Sigma(T)$
			\item<1-> $\Sigma(T)$ is the set of variables in $T$
			\item<2-> An assignment $\alpha$ is called a model of $T$ iff it is complete and satisfies
				all clauses of $T$
		\end{itemize}
	\end{frame}

	\subsection{Symmetry}
	\begin{frame}
		\frametitle{Symmetry}

		\begin{definition}[Symmetry]
			Consider a permutation $\sigma$ of a theory $T$. $\sigma$ is called a symmetry if and only if:
			\begin{itemize}
				\item $\sigma(\neg x) = \neg(\sigma(x))$
				\item $\sigma(\alpha)$ is a model for $T$ iff $\alpha$ is a model for $T$
			\end{itemize}
		\end{definition}

		\pause

		\begin{proposition}
			Let $\sigma$ be a symmetry of a theory $T$ and $c$ a clause.
			\begin{equation}
				T \vdash c \quad \implies \quad T \vdash \sigma( c )
			\end{equation}
		\end{proposition}
	\end{frame}

	\subsection{Symmetry Breaking}
	\begin{frame}
		\frametitle{Symmetry Breaking}

		\begin{itemize}
			\item Static Symmetry Breaking
			\item Dynamic Symmetry Breaking
		\end{itemize}
	\end{frame}

	\subsection{Dynamic Symmetry Breaking}
	\begin{frame}
		\frametitle{Dynamic Symmetry Breaking}

		\begin{block}{Approach}
			Propagate symmetrical literals.

			Given theory $T$, partial assignment $\alpha$ and symmetry
			\alert<2->{$\sigma$ of $T \cup \alpha$}.

			For each literal $l$:
			\begin{equation}
				T \cup \alpha \vdash l \quad
				\implies
				\quad T \cup \alpha \vdash \sigma(l)
			\end{equation}

			Thus when $l$ is progated, $\sigma(l)$ can also be propagated.
		\end{block}
	\end{frame}

	\begin{frame}
		\frametitle{Dynamic Symmetry Breaking: Weak Activity}

		\begin{block}<+->{Weak Activity}
			Given a theory T, let $\alpha$ be the current assignment and $\delta$ the set of
			choice literals.
			A symmetry $\sigma$ of T is weakly active if $\sigma(\delta) \subseteq \alpha$
		\end{block}

		\begin{block}<+->{Inactivity}
			Given a theory T, let $\alpha$ be the current assignment and $\delta$ the set of
			choice literals.
			A symmetry $\sigma$ of T is weakly-inactive if $\exists l \in \sigma(\delta), l \not\in \alpha$
		\end{block}
	\end{frame}

\section{Symmetry Optimizing Variable Branching}

	\subsection{The idea}
	\begin{frame}
		\frametitle{Symmetry Optimizing Variable Branching}

		\begin{block}{Optimization through literal selection}
			Improve dynamic symmetry propagation performance by choosing literals that promote
			symmetry weak-activity.
		\end{block}
	\end{frame}

	\subsection{Implemented Heuristics}
	\begin{frame}
		\frametitle{Symmetry Related Variable Branching Heuristics}

		\begin{block}{Literal optimization}
			\begin{enumerate}
				\item<+-> Get first $n$ undecided variables $V$ from the variable heap
				\item<+-> For each $v \in V$: perform lookahead on choice literals $v$ and $\neg v$
				\item<+-> Continue with branch that has the most weakly-active symmetries
			\end{enumerate}
		\end{block}

		\visible<+->{
		\begin{block}{Approximation technique}
			Assume current choice literals $\delta$, next choice literal $l$, and assignment $\alpha$.
			Instead of performing the lookahead for $l$, let $\alpha' = \alpha \cup \{l\}$ and
			evaluate the number of active symmetries against that assignment.
		\end{block}
		}

	\end{frame}
	
	\begin{frame}
		\frametitle{Symmetry Related Variable Branching Heuristics}

		\begin{block}{Symmetry Counting}
			\dots
		\end{block}

		\visible<+->{
		\begin{block}{Symmetry Usage}
			\ldots
		\end{block}
		}

	\end{frame}

\section{Results}

	\subsection{Test Procedures}
	\begin{frame}
		\ldots
	\end{frame}
	
	\subsection{Hypothesis 1}
	\begin{frame}
		\begin{block}{Hypothesis 1}
			\label{hyp:increased_activity}
			Heuristic $H$ manages to keep more symmetries weakly active during search.
		\end{block}
	\end{frame}
	
	\begin{frame}
		\frametitle{Results}
		\ldots
	\end{frame}
	
	\subsection{Hypothesis 2}
	\begin{frame}
		\begin{block}{Hypothesis 2}
			\label{hyp:activity_equals_speed}
			When more symmetries are active during search, more literals can be propagated, resulting in
			better solve performance on average.
		\end{block}
	\end{frame}
	
	\begin{frame}
		\frametitle{Results}
		\ldots
	\end{frame}
	
	
\section{Break it down}

	\subsection{Conclusion}
	\begin{frame}
		\frametitle{Conclusion}
		\ldots
	\end{frame}

	\subsection{References}
	\begin{frame}[allowframebreaks]
		\frametitle{References}
		\bibliographystyle{IEEEtran}
		\bibliography{../bibliography}
	\end{frame}

	\begin{frame}
	\Huge{\centerline{Questions?}}
	\end{frame}

\end{document}
