Symmetry breaking has become an interesting method for increasing the solving speed of
certain Conjunctive Normal Form (CNF) Satisfiability (SAT) problems.
This is typically done by adding Symmetry Breaking Predicates (SBPs) to the SAT
instance~\cite{sakallah2009symmetry}, resulting in a more restricted search space.
For symmetry breaking to be applied effectively the given problem has to exert symmetrical
properties, i.e. some search tree is virtually identical to another.
Symmetry breaking has been shown to provide significant
speed-ups~\cite{darga2004exploiting,aloul2003solving}, and do not slow down the algorithm
when no obvious symmetries are present.

Conflict Driven Clause Learning (CDCL) solvers~\cite{marques1999grasp} choose in each iteration a value for an undefined variable.
When a conflict arises, because both a literal $l$ and $\neg l$ are chosen, the solver backtracks
and saves the cause of the conflict in a clause to prevent further occurrences.
Deciding which variable to branch on is an important aspect of a SAT-solver and is a common target
for domain specific optimization~\cite{een2004extensible}.
The research is focused on finding rules or heuristics that favour choosing literals
that optimize the number of applicable symmetries in a conflict-driven sat-solver.

There are generally speaking two types of symmetry breaking: Static symmetry breaking and
dynamic symmetry breaking.
We will investigate the possibility of improving a dynamic symmetry breaker described by
\cite{devriendt2012symmetry}.
Their paper describes a symmetry breaker based on propagating symmetrical literals.
The solver therefore keeps track of symmetries that are applicable under the current assignment.
More applicable symmetries implies more propagated literals, which would suggest that
optimizing the solver for the number of applicable symmetries would yield improved results.
A step in this direction has been investigated by \cite{devriendt2012symmetry} and yielded
promising speedups.
