
The process of choosing the most optimal branching literal is a NP-hard problem, hence it is not
possible to
just calculate the best solution. So instead it is necessary to use a heuristic.
%TODO WUT?
%These heuristics calculate a subset of the complete solution and base their decision on this subset.

In order to solve a SAT problem as efficient as possible the search space must be reduced as much as
possible.
In a dynamic symmetry breaker, reducing the search space
is achieved by taking symmetries into account while evaluating a SAT instance.
To reduce the search space as much as possible, heuristics could attempt to optimize the amount of
active symmetries as observed by~\cite{devriendt2012symmetry} in order to benefit from them
maximally in the propagation phase.

In this section we consider several implementations of this heuristic: promoting choice literals
that keep or make symmetries weakly active during the search process.
In the next section we propose a few heuristics and compare their theoretical runtime performance
and discuss their applicability.

\subsection{SA: Literal Lookahead}
	We propose a heuristic that uses \emph{literal lookahead} that directly optimizes the number of
	active symmetries after choosing a new literal.
	Given the definition of weak activity in \cite{devriendt2012symmetry} we can define
	\cref{cor:makes_inactive}.\\

	\begin{corollary}
		\label{cor:makes_inactive}
		Given a symmetry $\sigma$ and a choice literal $l$,
		$\sigma$ is made weakly inactive by choosing $l$ iff
		$\sigma$ is weakly active, and $\sigma(l) \notin \alpha$.
		Where $(\alpha,\delta,\textit{expl})$ is the solver state \textbf{after} unit propagation of $l$.
	\end{corollary}

	In order to determine the activity of symmetries exactly after choosing a literal,
	it is necessary to perform the unit propagation step under the assumption that the literal is chosen.
	Performing a lookahead is an expensive operation as it requires the solver to perform half an
	iteration and backtrack.

	In fact, it is necessary to perform $|L|$ lookaheads, where $L = \bar{\Sigma}(T)$ is the set of
	literals that do not yet have an assignment.
	And each lookahead consists of a unit propagation and a symmetry propagation step, followed by
	backtracking over that choice.
	The unit propagation phase iterates through all clauses in a time complexity of $O(|T|)$.
	This occurs in the worst case $O(|L|)$ times,
	once for each literal if all literals are assigned in in this unit propagation phase.
	Then comes the symmetry propagation phase, which iterates through all symmetries in $O(|\mathcal{S}|)$.
	If the symmetry propagation phase found a (weakly) active symmetry and propagated that,
	the propagation will repeat.
	In total this process can repeat more than $O(|L|)$ times,
	so the total worst-case complexity of the lookahead can be simplified to $O(|L|(|T| + |\mathcal{S}|))$.

	It is thus not feasible in general to perform this procedure for every undecided variable and
	select the optimal one.
	Several alternative approaches can be considered.
	One choice is to perform a lookahead on the choice variable $v$ that the solver
	selects based on variable activity to determine the preferred assignment ($v$ or $\neg v$).

	Additionally, one could perform this lookahead for the $n$ most eligible variables
	and select the one that makes the least symmetries inactive.
	By combining this with other heuristics on selecting the most eligible variables (using variable
	activity), a much faster runtime may be achieved.
	We chose this approach and denote this heuristic simply by the abbreviation SA (Symmetry
	Activity).

	%\subsubsection{Working around the lookahead} % does not look like a header
\subsection{SA-APPROX: Approximation of Literal Lookahead}
	It is also possible to approximate the amount of symmetries made inactive under the
	assignment that the propagation of $l$ will yield.
	Instead of propagating, we can look at the number of symmetries made inactive by $l$
	under the current assignment, that is: consider a symmetry $\sigma$ as
	made inactive by $l$ if it is weakly active under the current assignment $\alpha$ and
	$\sigma(l) \notin \alpha$.
	Note that this is a rough approximation only, as these symmetries could be made active by
	unit propagation after choosing $l$.

	Also note that it does have a much smaller performance footprint, as it does not need to execute
	a propagation and rollback.
	One only needs to notify the symmetries of the choice literal and loop over check if they are
	still active after that.
	Afterwards one would need to undo that.
	This method thus has a performance linear in the number of symmetries $|S|$.

\subsection{SC: Symmetry Counting}

	Choosing a literal that occurs in a symmetry, increases the activity of that symmetry.
	Following the line of thought of attempting to keep as much symmetries active as possible,
	one could attempt to promote literals that appear in many symmetries.

	A possible implementation of this heuristic would increase the activity for every variable in
	every symmetry, when the symmetries are inputted into the solver.
	This implementation would only cost a polynomial amount of runtime, at the initialization of the
	solver and can be combined with any other heuristic that honors the order of the variable heap.

\subsection{SU: Symmetry Usage}
	Symmetry Usage is a dynamical symmetry breaking heuristic
	which takes into account the number of times a symmetry has previously been used in the solver tree.
	If a symmetry is used often when solving the tree, we suppose that it is more important than other symmetries.

	We can implement the same approach as in the symmetry counting heuristic: when a symmetry is
	used to propagate literals, the variables that are part of that symmetry are bumped up the
	variable heap.

	% -- old implementation
	%When choosing a literal, those that are used in commonly used symmetries, are chosen first, such
	%that these \emph{important symmetries} will become more strongly active.

	%The implementation is a simple counter that is increased by 1 for each usage of the symmetry.
	%The main runtime drawback of this heuristic, is looping through all symmetries a literal is involved in.
	%We do this for each literal, which has a runtime of $O(|L|)$.
	%The other additional operations are of a constant time complexity $O(1)$, thus the additional overhead is $O(s|L|)$,
	%where $s$ is the maximum number of symmetries a literal is involved in.

