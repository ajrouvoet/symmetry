\todoi{
	Construct an argument to support the use of heuristics to find a variable that optimizes the
	amount of active symmetries
}

The authors of \cite{katebi2010symmetry} already suggest the possibility of optimizing the solver by
branching on literals that optimize the number of symmetries that are weakly active.
The took a step in that direction by lowering the position of variables $v$ on the variable heap for
every symmetry $\sigma$ such that $sigma(v) = \neg v$.
Because choosing such variables would cause these symmetries to become permanently inactive.
Using this heuristic they showed a considerable performance boost, which prompted our research.

In this section we consider several implementations of this heuristic: promoting choice literals
that keep or make symmetries weakly active.
In the next section we compare the performance of these implementations and discuss their
applicability.

\subsection{Literal Lookahead}
	Given the definition of weak activity in \todo{ref} we can deduce corollary
	\ref{cor:makes_inactive}. \\

	\begin{corollary}
		\label{cor:makes_inactive}
		Given a symmetry $\sigma$ and a literal $l$, $\sigma$ is made inactive by choosing $l$ iff
		$\sigma$ is weakly active, and $\sigma(l) \notin \alpha$. Where
		$(\alpha,\delta,\textit{expl})$ is the solver state \emph{after} propagation of $l$.
	\end{corollary}

	In order to determine the activity of symmetries exactly after choosing a literal, it is thus
	necessary to perform the unit propagation step under the assumption that the literal is true.
	Performing a lookahead is an expensive operation as it requires the solver to perform half an
	iteration and backtrack.
	It is thus not feasible in general to perform this procedure for every undecided variable and
	select the optimal one.
	Additionally, this approach would disable all other heuristics for variable selection.

	Several alternative approaches can be considered:

	\begin{enumerate}
		\item An obvious option is to perform a lookahead on the choice variable $v$ that the solver
			selects (based on the variable heap and activity\todo{ref}) to determine the
			preferred assignment ($v$ or $\neg v$).

		\item As an alternative one could perform this lookahead for the first $n$ variables on the
			heap.

		\item One could attempt to approximate the amount of symmetries made inactive under the
			assignment that the propagation of $l$ will yield, by the number of symmetries made
			inactive by $l$ under the current assignment, that is: consider a symmetry $\sigma$ as
			made inactive by $l$ if it is weakly-active under the current assignment $\alpha$ and
			$\sigma(l) \notin \alpha$.
			Note that this is an approximation only, as these symmetries could be made active by
			unit propagation after choosing $l$.
			\todoi{give arguments to support this approximation}

	\end{enumerate}
