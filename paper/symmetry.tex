\documentclass[11pt, a4paper]{IEEEtran}

	\usepackage[english]{babel}
	\usepackage{amsmath}
	\usepackage{amsthm}
	\usepackage{amssymb}
	\usepackage{hyperref}
	\usepackage{graphicx}
	\usepackage{fullpage}	%Smaller margins
	\usepackage{parskip}

	% authors and affiliations
	\usepackage{authblk}
	\renewcommand\Authands{ and } %Remove weird comma before and

	%Internal references to section, images, etc. use \cref{}
	\usepackage[nameinlink,capitalise]{cleveref}

	%Link colors
	\usepackage{xcolor}
	\definecolor{dark-red}{rgb}{0.4,0.15,0.15}
	\definecolor{dark-blue}{rgb}{0.15,0.15,0.4}
	\definecolor{medium-blue}{rgb}{0,0,0.5}
	\hypersetup{
		colorlinks, linkcolor={dark-blue},
		citecolor={dark-blue}, urlcolor={medium-blue}
	}

	\newtheorem{theorem}{Theorem}
	\newtheorem{definition}{Def.}
	\newtheorem{proposition}{Prop.}
	\newtheorem{corollary}{Corollary}

	%Todos
	\usepackage{todonotes}
	\newcommand{\todoi}{\todo[inline]}

	\author{J. van Beek}
	\author{A.J. Rouvoet}
	\author{E. Schoute}
	\author{I.J.G. in 't Veen}

	\affil{Algorithmics Group\\
		Delft University of Technology\\
		Mekelweg 4, 2628 CD Delft\\
		\small \{J.vanBeek,A.J.Rouvoet, E.Schoute, I.J.G.intVeen\}@student.tudelft.nl
	}


\title{Symmetry Breaking}

\begin{document}
	\maketitle

	%\tableofcontents
	\listoftodos

	\begin{abstract}
		\todoi{Abstract}
	\end{abstract}

	\section{Introduction}
	\label{sec:Introduction}

		Symmetry breaking has become an interesting method for increasing the solving speed of
		certain Conjunctive Normal Form (CNF) Satisfiability (SAT) problems.
		This is typically done by adding Symmetry Breaking Predicates (SBPs) to the SAT
		instance~\cite{sakallah2009symmetry}, resulting in a simpler search space.
		There are generally speaking two types of symmetry breaking: Static symmetry breaking and
		dynamic symmetry breaking.
		While static symmetry breaking has become more popular\todo{Need ref},
		we will investigate the pros and cons of dynamic symmetry breaking.

		We will also investigate what classes of problems symmetry breaking can be applied to and
		what the effective speed-up pertains.
		For symmetry breaking to be applied effectively the given problem has to exert symmetrical
		properties, i.e.\ some search tree is virtually identical to another.
		Symmetry breaking has been shown to provide significant
		speed-ups~\cite{darga2004exploiting,aloul2003solving}, and do not slow down the algorithm
		when no obvious symmetries are present.

		Then we will investigate these claims by applying a given symmetry breaking algorithm to a
		problem, modifying it and empirically judge its effectiveness. \todoi{Introduction to our SB
		algorithm}

	\section{Background} \label{sec:Background}
		To prevent a SAT solver from researching a symmetrical search tree, that was already solved
		before, Symmetry Breaking Predicates can be added to the SAT
		instance~\cite{sakallah2009symmetry}. SBPs can also be applied partially to the symmetry
		group, called \textit{partial symmetry breaking}~\cite{sakallah2009symmetry}. Because
		\textit{full symmetry breaking} sometimes costs a lot of time, most symmetrical search
		spaces can be blocked by using a subset of the full SBP. There has been more research on
		the efficiency of SBPs, such as \cite{aloul2006efficient}\todo{More refs}.

		\subsection{Terminology}
			As we build our research on the results of \cite{devriendt2012symmetry}, we will adopt
			their notation and terminology.

			Let \emph{a theory} $T$ be a SAT-instance in conjunctive normal form, i.e. a conjuction
			of clauses.
			Let $\Sigma(T)$ denote the set of variables in $T$.
			Every clause is a disjunction of \emph{literals}, where a literal is either of the form
			$x$ or $\neg x$, where $x \in \Sigma(T)$.
			$\bar\Sigma(T)$ denotes the set of all possible literals of $T$.

			An \emph{assignment} of $T$ is an $\alpha \subseteq \bar\Sigma(T)$, such that at most
			one literal for each variable in $T$ appears in $\alpha$.
			\emph{Partial} and \emph{complete} assignments are distinguished.
			An assignment is \emph{complete} if it contains one literal for each $x \in \Sigma(T)$.

			A literal $l$ can be evaluated against a assignment $\alpha$: $l$ is \emph{true} if $l \in
			\alpha$, \emph{false} if $\neg l \in \alpha$ and \emph{undefined} otherwise.
			Clauses are either \emph{satisfied} or \emph{in conflict} with an assignment $\alpha$.
			If a clause $C$ contains only false literals, then it is considered \emph{in conflict}.
			If $C$ contains at least one true literal, it is satisfied.
			Additionally $C$ is a \emph{unit-clause} if it only contains one true literal.

			A SAT-solver looks for a complete assignment such that all clauses in $T$ are satisfied.
			Such an assignment is called a model for $T$.
			A theory of $T$ is satisfiable if a model exists for $T$.
			Given a theory $T$, we consider a clause $c$ a \emph{logical consequence} of $T$ if
			$T \wedge c)$ is satisfied under every model of $T$; this is denoted as $T \vdash c$.
			A theory $T'$ is a logical consequence of $T$, if all clauses in $T$ are logical
			consequences of $T$.

			\todoi{This is almost directly taken from \cite{devriendt2012symmetry}. Is that a
			problem?}

			\todoi{Conflict-Driven symmetry breaking}

				Consider a permutation $\sigma$ of a theory $T$.
				$\sigma$ is called a symmetry if and only if:

				\begin{enumerate}
					\item $\sigma(\neg x) = \neg(\sigma( x ))$
					\item $\sigma(\alpha )$ is a model for $T$ if and only if $\alpha$ is a model for
						$T$
				\end{enumerate}

				A crucial theorem concerning symmetries is the following: \\

				\begin{proposition}
					\label{prop:symmetric_clause_learning}
					Let $s$ be a symmetry of a theory $T$ and $c$ a clause.
					\begin{equation}
						T \vdash c \quad \implies \quad T \vdash \sigma( c )
					\end{equation}
				\end{proposition}

		\subsection{Static Symmetry Breaking}
			Static symmetry breaking algorithms modify the input of the SAT solver, but do not
			adjust the solver itself~\cite{sakallah2009symmetry}.

		\subsection{Dynamic Symmetry Breaking}
			Dynamic symmetry breaking algorithms modify the SAT solver to search for local
			symmetries during runtime. Contributions to dynamic symmetry breaking have not been
			numerous in SAT such as \cite{sabharwal2005symchaff}, where Constraint Satisfaction
			Problems (CSP) have looked more research in the matter.

			One theoretical approach in dynamic symmetry breaking would be to exploit symmetries in
			a theory to learn more clauses.
			Every time the solver learns a clause $c$, $\sigma(c)$ could also be learned, according
			to proposition \ref{prop:symmetric_clause_learning}.
			This however is often not feasible, as it would add too many clauses to the theory.

			The alternative approach, which is taken by the authors of
			\cite{devriendt2012symmetry}, is to propagate symmetrical literals.
			The idea is that given a theory $T$, a partial assignment $\alpha$ and a symmetry
			$\sigma$ of $T \cup \alpha$, proposition \ref{prop:symmetric_clause_learning} implies
			that for any literal $l$:

			\begin{equation}
				T \cup \alpha \vdash l \quad
				\implies
				\quad T \cup \alpha \vdash \sigma(l)
			\end{equation}

			Thus, when the solver propagates a literal $l$, $\sigma(l)$ can also be propagated.
			\todo{what about $\sigma^2(l)$\ldots}.

			The challenge with this approach is to find symmetries of $T \cup \alpha$ in a
			reasonable time.
			To this end, the authors of \cite{devriendt2012symmetry} introduced the notion of
			\emph{weak activity}.
			Weakly active symmetries can be used to propagate symmetrical literals.

	% \nocite{*} %Show everything in bib
	\newpage
	\bibliographystyle{IEEEtran}
	\bibliography{../bibliography}
\end{document}
