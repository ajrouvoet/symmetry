\documentclass[11pt, a4paper, abstracton]{scrartcl}

	\usepackage[english]{babel}
	\usepackage{amsmath}
	\usepackage{amsthm}
	\usepackage{amssymb}
	\usepackage{hyperref}
	\usepackage{graphicx}
	\usepackage{fullpage}	%Smaller margins
	\usepackage{parskip}

	% authors and affiliations
	\usepackage{authblk}
	\renewcommand\Authands{ and } %Remove weird comma before and
	
	%Internal references to section, images, etc. use \cref{}
	\usepackage[nameinlink,capitalise]{cleveref}
	
	%Link colors
	\usepackage{xcolor}
	\definecolor{dark-red}{rgb}{0.4,0.15,0.15}
	\definecolor{dark-blue}{rgb}{0.15,0.15,0.4}
	\definecolor{medium-blue}{rgb}{0,0,0.5}
	\hypersetup{
		colorlinks, linkcolor={dark-blue},
		citecolor={dark-blue}, urlcolor={medium-blue}
	}

	%Todos
	\usepackage{todonotes}
	\newcommand{\todoi}{\todo[inline]}

	\author{J. van Beek}
	\author{A.J. Rouvoet}
	\author{E. Schoute}
	\author{I.J.G. in 't Veen}

	\affil{Algorithmics Group\\
		Delft University of Technology\\
		Mekelweg 4, 2628 CD Delft\\
		\small \{J.vanBeek,A.J.Rouvoet, E.Schoute, I.J.G.intVeen\}@student.tudelft.nl
	}


\title{Symmetry Breaking}

\begin{document}
	\maketitle
	
	%\tableofcontents
	\listoftodos

	\begin{abstract}
		\todoi{Abstract}
	\end{abstract}
	
	\section{Introduction} \label{sec:Introduction}
		Symmetry breaking has become an interesting method for increasing the solving speed of certain Conjunctive Normal Form (CNF) Satisfiability (SAT) problems.
		This is typically done by adding Symmetry Breaking Predicates (SBPs) to the SAT instance~\cite{sakallah2009symmetry},
		resulting in a simpler search space.
		There are generally speaking two types of symmetry breaking: Static symmetry breaking and dynamic symmetry breaking.
		While static symmetry breaking has become more popular\todo{Need ref}, we will investigate the pros and cons of dynamic symmetry breaking.
		
		We will also investigate what classes of problems symmetry breaking can be applied to and what the effective speed-up pertains.
		For symmetry breaking to be applied effectively the given problem has to exert symmetrical properties,
		i.e.\ some search tree is virtually identical to another.
		Symmetry breaking has been shown to provide significant speed-ups~\cite{darga2004exploiting,aloul2003solving},
		and do not slow down the algorithm when no obvious symmetries are present.
		
		Then we will investigate these claims by applying a given symmetry breaking algorithm to a problem, modifying it and empirically judge its effectiveness. \todoi{Introduction to our SB algorithm}
		
	\section{Background} \label{sec:Background}
		Symmetry Breaking Predicates...
		
		\subsection{Static Symmetry Breaking}
			\todoi{Static SB}
			
		\subsection{Dynamic Symmetry Breaking}
			\todoi{Dynamic SB}
		

	\nocite{*} %Show everything in bib
	\bibliographystyle{IEEEtran}
	\bibliography{../bibliography}
\end{document}
