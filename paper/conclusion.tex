We have looked at a recently proposed technique for symmetry propagation by Devriendt et al.~\cite{devriendt2012symmetry}.
After analysing their method, improvements and heuristics we conjectured that
keeping more symmetries active during solving would result in better performance.
Assuming this hypothesis the heuristics we've proposed should show to have a better performance than
the implementation of \cite{devriendt2012symmetry} without their heuristic on inverting symmetries
$SP^{REG}$.

The heuristics we proposed are Symmetry Activity Lookahead, $SA$, an approximation of this
heuristic, $SA-APPROX$, Symmetry Counting, $SC$, and Symmetry Usage, $SU$.
We then implemented the proposed heuristics and performed tests on a benchmark set also used
in~\cite{devriendt2012symmetry}.

The measurements indicate that SA, SA-APPROX and SC all succeeded to increase the number of weakly
active symmetries during search.
SU did not manage this and this was attributed to the fact that it cannot properly promote important
symmetries through variable activity.

From the results we can largely falsify the second hypothesis, i.e. optimizing the number of
active symmetries during search does not correlate with higher performance.
This is surprising, because we expected that with a lot of active symmetries the symmetry
propagation phase would yield more results, resulting in a more efficient search process.

However, we must note that we have not succeeded in fully explaining the complexity of the
different results (see Future Work, \cref{sec:FutureWork}).
