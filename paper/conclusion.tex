We have looked at a recently proposed technique for symmetry propagation by Devriendt et al.~\cite{devriendt2012symmetry}.
After analysing their method, improvements and heuristics we conjectured that
keeping more symmetries active during solving would result in better performance.
Assuming this hypothesis the heuristics we've proposed should show to have a better performance than
MiniSat without any heuristics, $SP^{OPT}$.

The heuristics we proposed are Literal Lookahead, $SA$, an approximation of this heuristic, $SA-APPROX$,
Symmetry Counting, $SC$, and Symmetry Usage, $SU$.
We then implemented the proposed heuristics and performed tests on a benchmark set also used in~\cite{devriendt2012symmetry}.
\todoi{Results}
From the results we can largely falsify the hypothesis,
so that we know that average symmetry activity does not give a higher performance.
This is surprising, because we expected that with a lot of symmetries the symmetry propagation phase would be more succesfull.
However, the results show that the extra runtime of the heuristics
that improve the number of active symmetries does not outweigh the benefits of more active symmetries.