To prevent a SAT solver from researching a symmetrical search tree, that was already solved
before, Symmetry Breaking Predicates can be added to the SAT
instance~\cite{sakallah2009symmetry}. SBPs can also be applied partially to the symmetry
group, called \textit{partial symmetry breaking}~\cite{sakallah2009symmetry}. Because
\textit{full symmetry breaking} sometimes costs a lot of time, most symmetrical search
spaces can be blocked by using a subset of the full SBP. There has been more research on
the efficiency of SBPs, such as~\cite{aloul2006efficient,aloul2003shatter,crawford1996symmetry} and
~\cite{sabharwal2005symchaff}.

\subsection{Terminology}
	As we build our research on the results of \cite{devriendt2012symmetry}, we will adopt
	their notation and terminology.
	Let a \emph{theory} $T$ be a SAT-instance in conjunctive normal form, i.e. a conjunction
	of clauses.
	Let $\Sigma(T)$ denote the set of variables in $T$.
	A literal is either of the form	$x$ or $\neg x$, where $x \in \Sigma(T)$.
	$\bar\Sigma(T)$ denotes the set of all possible literals of $T$.

	An \emph{assignment} of $T$ is an $\alpha \subseteq \bar\Sigma(T)$, such that at most
	one literal for each corresponding variable in $\Sigma(T)$ appears in $\alpha$.
	Thus an assignment where both $\{x, \neg x\} \subseteq \alpha$ is not allowed.
	\emph{Partial} and \emph{complete} assignments are distinguished.
	An assignment is considered \emph{complete} if it contains one literal for each $x \in \Sigma(T)$,
	otherwise it is a \emph{partial} assignment.

	A literal $l$ can be evaluated against an assignment $\alpha$: $l$ is \emph{true} if $l \in
	\alpha$, \emph{false} if $\neg l \in \alpha$ and \emph{undefined} otherwise.
	Clauses are either \emph{satisfied} or in \emph{conflict} with an assignment $\alpha$.
	A clause $C$ is \emph{satisfied} under the assignment $\alpha$,
	when at least one literal $l \in C$ is true.
	A \emph{conflict clause} $C$ is a clause which contains only literals that are false under the assignment.
	Additionally, $C$ can be a \emph{unit clause} if all literals are false, except one.

	A SAT-solver looks for a complete assignment such that all clauses in $T$ are satisfied.
	Such an assignment is called a model for $T$.
	A theory $T$ is satisfiable if a model exists for $T$.
	Given a theory $T$, we consider a clause $c$ a \emph{logical consequence} of $T$ iff
	$T \wedge c$ is satisfied for every model of $T$, this is denoted as $T \models c$.
	A theory $T'$ is a logical consequence of $T$, if all clauses in $T'$ are logical
	consequences of $T$.

	The SAT solver as used by \cite{devriendt2012symmetry} is called MiniSAT
	\cite{een2004extensible} and uses the principle of \emph{Conflict Driven Clause Learning}, hence
	it is a CDCL SAT solver \cite{zhang2001efficient}.
	For each step of the solver, it assigns a value to an unassigned variable and adds the
	corresponding literal to the assignment.
	Because of this assignment, some clauses may become unit clauses, in which a literal must be
	satisfied or a conflict clause will result.
	In the \emph{unit propagation phase} the solver assigns these \emph{propagated literals} $l$ to
	true and the unit clauses are added as \emph{explanation clauses}, $expl(l)$.
	Such a solver has as state a triplet $(\alpha, \delta, expl)$,
	where $\delta\subseteq \alpha$ are the \emph{choice literals} that are freely assigned between
	propagations.
	Such that $\alpha \setminus \delta$ are the \emph{propagated literals}.

	When at some point a literal $l$ would be propagated while $\neg l$ is already true under the
	current assignment, this causes a conflict.
	If a conflict arises the CDCL SAT solver constructs a \emph{learned clause} using the
	explanation clauses $expl(l)$ causing the conflict.
	By adding this learned clause $C$ to the theory $T \cup C$ the solver prevents them from
	occurring again, and because $T \models C$ it does not affect the set of satisfying assignments.

	Consider a permutation $\sigma$ of $\bar\Sigma(T)$, $\sigma$ is called a symmetry if and only
	if:
	\begin{enumerate}
		\item $\sigma(\neg x) = \neg(\sigma( x ))$
		\item $\sigma(\alpha )$ is a model for $T$ $\Leftrightarrow$ $\alpha$ is a model for $T$
	\end{enumerate}

	A crucial theorem concerning symmetries is that for any clause $c$ that is a logical consequence
	of the theory $T$, $\sigma(c)$ is also a logical consequence.
	This can be used in clause learning to also add symmetric clauses to the learned clauses.\\

	\begin{proposition}
		\label{prop:symmetric_clause_learning}
		Let $\sigma$ be a symmetry of a theory $T$ and $c$ a clause.
		\begin{equation}
			T \models c \quad \implies \quad T \models \sigma( c )
		\end{equation}
	\end{proposition}

\subsection{Static Symmetry Breaking}
	Static symmetry breaking algorithms modify the input of the SAT solver, but do not
	adjust the solver itself~\cite{sakallah2009symmetry}.

	These solvers work by adding additional symmetry breaking predicates to a CNF
	formula. Such a predicate acts like a filter, reducing the search space to mostly
	non-symmetric regions.
	The symmetry breaker does this without changing the satisfiability of the CNF formula~\cite{aloul2003shatter}.

	Symmetry breaking can only be effective if the overhead of calculating symmetries and
	adding the symmetry breaking predicates takes significantly less time then the time it
	saves of the problem solving process. In order to limit the time needed for the static
	symmetry breaking, and the effort needed by the solver when evaluating all these
	additional predicates, it is often feasible to not break \emph{all} but \emph{enough}
	symmetries as proposed by \cite{aloul2003shatter}.

	One, often used, static symmetry breaker is Shatter~\cite{aloul2003shatter}.
	This symmetry breaker can also be used as a preprocessor for a dynamic symmetry breaker.
	It is used to find the present symmetries in a given problem using Saucy 2.0~\cite{darga2004exploiting},
	which can then be used by the algorithm.
	Additionally, Shatter can be used to add static symmetry breaking predicates to the theory.

\subsection{Dynamic Symmetry Breaking}
	Dynamic symmetry breaking algorithms search for local symmetries during runtime.
	Contributions to dynamic symmetry breaking have not been numerous in SAT,
	e.g Symchaff~\cite{sabharwal2005symchaff} and Symmetry Propagation~\cite{devriendt2012symmetry}.
	In the field of Constraint Satisfaction	Problems (CSP) dynamic symmetry breaking has had more contributions.
	In theory the techniques developed in that field could be translated to the SAT domain.

	One theoretical approach in dynamic symmetry breaking would be to exploit symmetries in
	a theory to learn more clauses.
	Every time the solver learns a clause $c$, $\sigma(c)$ could also be learned, according
	to \cref{prop:symmetric_clause_learning}.
	This however is often not feasible, as it would add too many clauses to the theory.

	The alternative approach, which is proposed by Devriendt et al.~\cite{devriendt2012symmetry},
	is to propagate symmetrical literals and is discussed in the next section.

\subsection{Symmetry Propagation}
	The idea is that given a theory $T$, a partial assignment $\alpha$ and a symmetry
	$\sigma$ of $T \cup \alpha$, \cref{prop:symmetric_clause_learning} implies
	that for any literal $l$:\\

	\begin{proposition}
		\label{prop:symmetryPropagation}
		Let $\sigma$ be a symmetry of $T \cup \alpha$
		\begin{equation*}
			T \cup \alpha \models l \quad
			\implies
			\quad T \cup \alpha \models \sigma(l)
		\end{equation*}
	\end{proposition}
	Thus, when the solver propagates a literal $l$, $\sigma(l), \sigma^2(l),\dots$ can also be propagated.

	The challenge with this approach is to find symmetries of $T \cup \alpha$ in a
	reasonable time.
	To this end, the authors of \cite{devriendt2012symmetry} introduced the notion of
	\emph{weak activity}.
	Weakly active symmetries can be used to propagate symmetrical literals.
	Weak activity is defined as follows~\cite{devriendt2012symmetry}:
	\begin{definition}
		\label{def:weaklyActive}
		``Given a theory $T$, let $(\alpha, \delta, expl)$ be the state of a solver.
		A symmetry $\sigma$ is \textbf{weakly active} for  assignment $\alpha$ if
		$\sigma(\delta)\subseteq \alpha$.''
	\end{definition}
	In a few steps (see~\cite{devriendt2012symmetry}) we can derive
	\begin{equation}
		T\cup \alpha \models T \cup \sigma(\alpha) \models T \cup \sigma^2(\alpha) \models \dots.
	\end{equation}
	And it is shown that propagating the (weakly) active symmetries of choice literals is valid.

	The Symmetry Propagation (SP) algorithm as proposed by Devriendt et
	al.~\cite{devriendt2012symmetry} is a changed CDCL solver.
	The choice literal is chosen based on variable activity, after which as normal unit propagation
	occurs.
	Then, if possible, one unused symmetry is propagated, through the use of the \emph{first
	asymmetric literal}, which keeps track of which is the first choice literal in the assignment
	that is part of the symmetry.
	After \emph{symmetry propagation} unit propagation occurs again, alternating between the two,
	until it is no longer possible to propagate literals.
	Then another iteration begins with the choice of a literal, etc. until a satisfying assignment
	is found.
	It is possible to include an explanation for first asymmetric literals $l$ that were propagated
	in symmetry propagation,
	by adding $expl(\sigma(l)) = \sigma(expl(l))$ to the explanation store for symmetric literal
	$\sigma(l)$.

	Two versions are described: the basic implementation, denoted as $SP^{REG}$, and the optimized
	implementation $SP^{OPT}$.
	The optimized version contains a heuristic for variable branching based on inverting literals,
	which we describe in the following section.
	We will compare both to the optimized and the regular version.
	Comparison with the regular version will show whether or not the heuristic is working, whereas
	comparison with $SP^{OPT}$ will give a measure of how good it is performing.

\subsection{Variable Activity}
	The easiest strategy to choose a literal is a random choice.
	However, there are possibly better heuristics available for branching,
	such as BOHM's and MOM's heuristic~\cite{marques1999impact}.

	Because choosing variables to branch on is an important target for optimization in SAT solvers,
	and heuristics can complement each other, MiniSAT contains a mechanism that heuristics can
	exploit to give precedence to certain variables.
	The solver maintains a heap of variables, each of which has a activity.
	Variables can be put higher on the heap by a heuristic, and when a literal should be chosen by
	the solver, it will take the one with the highest activity.

	An interesting example of a heuristic that is implemented in MiniSAT is described by
	\cite{moskewicz2001chaff}: Variable State Independent Decaying Sum (VSIDS).
	It increases the activity of unassigned variables that have been chosen before.
	In effect, this makes the solver attempt to solve recent conflict clauses.
	They reason that because conflict clauses drive the search process in difficult problems,
	solving these clauses first will increase the solve speed.

	Another example is the optimization that is included in the optimized version of SP.
	The decrease the activity of \emph{inverting} literals on the variable heap.
	A literal $v$ is inverting under a symmetry $\sigma$ if $\sigma(v) = \neg v$.
	Choosing such variables would cause these symmetries to become permanently inactive, such that
	the could no longer be used to propagate symmetric literals.
	Using this heuristic they showed a considerable performance boost, which prompted our research.
