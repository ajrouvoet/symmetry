\documentclass[11pt, a4paper]{IEEEtran}

	\usepackage[english]{babel}
	\usepackage{amsmath}
	\usepackage{amsthm}
	\usepackage{amssymb}
	\usepackage{hyperref}
	\usepackage{graphicx}
	\usepackage{fullpage}	%Smaller margins
	\usepackage{parskip}

	% authors and affiliations
	\usepackage{authblk}
	\renewcommand\Authands{ and } %Remove weird comma before and

	%Internal references to section, images, etc. use \cref{}
	\usepackage[nameinlink,capitalise]{cleveref}

	%Link colors
	\usepackage{xcolor}
	\definecolor{dark-red}{rgb}{0.4,0.15,0.15}
	\definecolor{dark-blue}{rgb}{0.15,0.15,0.4}
	\definecolor{medium-blue}{rgb}{0,0,0.5}
	\hypersetup{
		colorlinks, linkcolor={dark-blue},
		citecolor={dark-blue}, urlcolor={medium-blue}
	}

	\newtheorem{theorem}{Theorem}
	\newtheorem{definition}{Def.}
	\newtheorem{proposition}{Prop.}
	\newtheorem{corollary}{Corollary}

	%Todos
	\usepackage{todonotes}
	\newcommand{\todoi}{\todo[inline]}

	\author{J. van Beek}
	\author{A.J. Rouvoet}
	\author{E. Schoute}
	\author{I.J.G. in 't Veen}

	\affil{Algorithmics Group\\
		Delft University of Technology\\
		Mekelweg 4, 2628 CD Delft\\
		\small \{J.vanBeek,A.J.Rouvoet, E.Schoute, I.J.G.intVeen\}@student.tudelft.nl
	}


	\title{Research Proposal:}
	\author{
		A.J. Rouvoet (a.j.rouvoet@student.tudelft.nl) \\ 4036964 \and
		\ldots
	}

	\subtitle{Symmetry Breaking in SAT}

\begin{document}

	\maketitle
	\newpage

	\section{Research}
	As the subtitle suggests, our research focusses on symmetry breaking in satisfiability solvers.

	We will study existing material on dynamic and static symmetry breaking and attempt to implement such a dynamic method in minisat.

		\subsection{Research Questions}
		What are the advantages and complications of an implementation of a generic dynamic	symmetry breaking algorithm in a SAT solver, compared to a static symmetry breaking implementation?

		\subsection{Motivation}
			As a research group we are focused on getting a better understanding of SAT solvers and methods of exploiting symmetry in SAT instances for improving the speed by which a SAT	instance can be solved.	Writing this paper is part of the course IN4077, which has as it's main objectives:

			\begin{enumerate}
				\item
					Apply combinatorial solvers to problem domains, and can compare and evaluate them.
				\item
					Explain general techniques used in combinatorial solvers
				\item
					Design and implement an extension or modification of a solver in order to answer the research question.
			\end{enumerate}

			Besides these goals, that we expect to be fulfilled by studying the material necessary to answer the formulated research question, there are a few additional goals formulated for the course:

			\begin{enumerate}
				\item Formulate a research question on a provided topic.
				\item Communicate his/her findings effectively.
			\end{enumerate}

			These goals are fulfilled respectively by this document and the planned paper and the	presentation on our findings.

	\section{Context}
	There are two types of symmetry breaking, static and dynamic. In this research project we will focus mostly on the dynamic variant. Dynamic symmetry breaking involves interacting with the solver during search to try to avoid symmetrical search spaces that have already been visited. Static symmetry breaking, on the other hand, involves adding extra constraints to the input to avoid symmetrical search spaces, before it is fed to the solver.

	\section{Research Plan}

		\subsection{Activities}

		\subsection{Expected Results}

	\section{Conclusions}

	% \bibliographystyle{IEEEtran}
	% \bibliography{../bibliography}

\end{document}
