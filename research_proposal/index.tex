\documentclass[a4paper]{scrartcl}

	\usepackage{todonotes}
	\usepackage{parskip}

	\title{Research Proposal:}
	\author{
		A.J. Rouvoet (a.j.rouvoet@student.tudelft.nl) \\ 4036964 \and
		\ldots
	}

	\subtitle{Symmetry Breaking in SAT}

\begin{document}

	\maketitle
	\newpage

	\section{Research}
		As the subtitle suggests, our research focusses on symmetry breaking in satisfiability
		solvers.

		We will study existing material on dynamic and static symmetry breaking and attempt to
		implement such a dynamic method in minisat.

		\subsection{Research Questions}
			What are the advantages and complications of an implementation of a generic dynamic
			symmetry breaking algorithm in a SAT solver, compared to a static symmetry breaking
			implementation?

		\subsection{Motivation}
			As a research group we are focused on getting a better understanding of SAT solvers and
			methods of exploiting symmetry in SAT instances for improving the speed by which a SAT
			instance can be solved.
			Writing this paper is part of the course IN4077, which has as it's main objectives:

			\begin{enumerate}
				\item
					Apply combinatorial solvers to problem domains, and can compare and evaluate
					them.
				\item
					Explain general techniques used in combinatorial solvers
				\item
					Design and implement an extension or modification of a solver in order to answer
					the research question.
			\end{enumerate}

			Besides these goals, that we expect to be fulfilled by studying the material necessary
			to answer the formulated research question, there are a few additional goals formulated
			for the course:

			\begin{enumerate}
				\item Formulate a research question on a provided topic.
				\item Communicate his/her findings effectively.
			\end{enumerate}

			These goals are fulfilled respectively by this document and the planned paper and the
			presentation on our findings.

	\section{Context}
		\todo{Provide the context of the research: i.e. what the heck are SAT solvers, and what do
		we mean by symmetry and symmetry breaking? What is  the difference between static and
		dynamic?}

	\section{Research Plan}

		\subsection{Activities}

		\subsection{Expected Results}

	\section{Conclusions}

	% \bibliographystyle{IEEEtran}
	% \bibliography{../bibliography}

\end{document}
